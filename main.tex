\documentclass{beamer}
\usetheme[secheader]{Boadilla}

\defbeamertemplate*{footline}{Boadilla}
{
    \leavevmode%
    \hbox{%
%   \begin{beamercolorbox}[wd=.333333\paperwidth,ht=2.25ex,dp=1ex,right]{author in head/foot}%
%     \usebeamerfont{author in head/foot}\insertshortauthor~~(\insertshortinstitute)
%   \end{beamercolorbox}%
%   \begin{beamercolorbox}[wd=.333333\paperwidth,ht=2.25ex,dp=1ex,right]{title in head/foot}%
%     \usebeamerfont{title in head/foot}\insertshorttitle
%   \end{beamercolorbox}%
        \begin{beamercolorbox}[wd=1\paperwidth,ht=2.25ex,dp=1ex,right]{date in head/foot}%
%     \usebeamerfont{date in head/foot}\insertshortdate{}\hspace*{2em}
            \insertframenumber{}\hspace*{2ex}
        \end{beamercolorbox}}%
    \vskip0pt%
}


\usepackage{comment}
% wow this is a hack that lets you have more definitions in latex.
% see http://www.tex.ac.uk/cgi-bin/texfaq2html?label=noroom
\usepackage{natbib}
\usepackage{etex}
\usepackage{pgffor}
\usepackage[utf8]{inputenc}
%\usepackage[cyr]{aeguill}
\reserveinserts{28}
\usepackage{tikz}
\usepackage[customcolors]{hf-tikz}
%\usepackage[utf8]{inputenc}
%\mode<presentation>{\usetheme{Caltech}}

\usepackage{amsmath}
\usepackage{mathtools}
\usepackage{amssymb}
\usepackage{amsfonts}
\usepackage{amsthm}
\usepackage{multimedia}
\usepackage{color}
\usepackage{esint}
\usepackage{stmaryrd}
\usepackage{tabularx}
\usepackage{multirow}
\usepackage[squaren]{SIunits}
\usepackage{graphicx}
\usepackage{diagbox}
\usepackage{pdfpages}
\usepackage{dsfont}
\usepackage{xcolor}
\usepackage{soul}
\usepackage[linesnumbered,ruled,vlined]{algorithm2e}

\usepackage{subfig}

\graphicspath{{./img/}}

\newcommand{\cplus}{\colorbox{green}{($+$)} }
\newcommand{\cmoins}{\colorbox{red}{($-$)} }
\newcommand{\cmean}{\colorbox{yellow}{($\pm$)}}

\newcommand{\mathcolorbox}[2]{\colorbox{#1}{$\displaystyle #2$}}



%%%%%%%%%%%%%%%%%%%%%%%%%%%%
% Paper dependent stuff    %
%%%%%%%%%%%%%%%%%%%%%%%%%%%%


\newcommand{\ov}{\overline}
\newcommand{\oa}{\ov{a}}
%\newcommand{\oQ}{\ov{Q}}
%\newcommand{\oR}{\ov{R}}
\newcommand{\ox}{\ov{x}}
\newcommand{\oz}{\ov{z}}
\newcommand{\oy}{\ov{y}}
\newcommand{\os}{\ov{s}}
%\newcommand{\or}{\ov{r}}
%\newcommand{\ocS}{\ov{\cS}}
\newcommand{\ocF}{\ov{\cF}}
%\newcommand{\augmentedtransition}{\ov{\transition}
%\newcommand{\ocA}{\ov{\cA}}

%%%%%%%%%%%%%%%%%%%%%%%%%%%%rans
% Aesthetics               %
% over-underline, hat, bold%
%%%%%%%%%%%%%%%%%%%%%%%%%%%%

\newcommand{\eps}{\varepsilon}
\newcommand{\vareps}{\varepsilon}
\renewcommand{\epsilon}{\varepsilon}
%\renewcommand{\hat}{\widehat}
\renewcommand{\tilde}{\widetilde}
\renewcommand{\bar}{\overline}

\newcommand*{\MyDef}{\mathrm{\scriptscriptstyle def}}
\newcommand*{\eqdefU}{\ensuremath{\mathop{\overset{\MyDef}{=}}}}% Unscaled version
\newcommand*{\eqdef}{\mathop{\overset{\MyDef}{\resizebox{\widthof{\eqdefU}}{\heightof{=}}{=}}}}


\def\:#1{\protect \ifmmode {\mathbf{#1}} \else {\textbf{#1}} \fi}
\newcommand{\CommaBin}{\mathbin{\raisebox{0.5ex}{,}}}

\newcommand{\wt}[1]{\widetilde{#1}}
\newcommand{\wh}[1]{\widehat{#1}}
\newcommand{\wo}[1]{\overline{#1}}
\newcommand{\wb}[1]{\overline{#1}}

% bf and bm missing due to conflict!!
\newcommand{\bsym}[1]{\mathbf{#1}}
\newcommand{\bzero}{\mathbf{0}}
\newcommand{\ba}{\mathbf{a}}
\newcommand{\bb}{\mathbf{b}}
\newcommand{\bc}{\mathbf{c}}
\newcommand{\bd}{\mathbf{d}}
\newcommand{\be}{\mathbf{e}}
\newcommand{\bg}{\mathbf{g}}
\newcommand{\bh}{\mathbf{h}}
\newcommand{\bi}{\mathbf{i}}
\newcommand{\bj}{\mathbf{j}}
\newcommand{\bk}{\mathbf{k}}
\newcommand{\bl}{\mathbf{l}}
\newcommand{\bn}{\mathbf{n}}
%\newcommand{\bo}{\mathbf{o}}
\newcommand{\bp}{\mathbf{p}}
\newcommand{\bq}{\mathbf{q}}
\newcommand{\br}{\mathbf{r}}
\newcommand{\bs}{\mathbf{s}}
\newcommand{\bt}{\mathbf{t}}
\newcommand{\bu}{\mathbf{u}}
\newcommand{\bv}{\mathbf{v}}
\newcommand{\bw}{\mathbf{w}}
\newcommand{\bx}{\mathbf{x}}
\newcommand{\by}{\mathbf{y}}
\newcommand{\bz}{\mathbf{z}}

\newcommand{\bA}{\mathbf{A}}
\newcommand{\bB}{\mathbf{B}}
\newcommand{\bC}{\mathbf{C}}
\newcommand{\bD}{\mathbf{D}}
\newcommand{\bE}{\mathbf{E}}
\newcommand{\bF}{\mathbf{F}}
\newcommand{\bG}{\mathbf{G}}
\newcommand{\bH}{\mathbf{H}}
\newcommand{\bI}{\mathbf{I}}
\newcommand{\bJ}{\mathbf{J}}
\newcommand{\bK}{\mathbf{K}}
\newcommand{\bL}{\mathbf{L}}
\newcommand{\bM}{\mathbf{M}}
\newcommand{\bN}{\mathbf{N}}
\newcommand{\bO}{\mathbf{O}}
\newcommand{\bP}{\mathbf{P}}
\newcommand{\bQ}{\mathbf{Q}}
\newcommand{\bR}{\mathbf{R}}
\newcommand{\bS}{\mathbf{S}}
\newcommand{\bT}{\mathbf{T}}
\newcommand{\bU}{\mathbf{U}}
\newcommand{\bV}{\mathbf{V}}
\newcommand{\bW}{\mathbf{W}}
\newcommand{\bX}{\mathbf{X}}
\newcommand{\bY}{\mathbf{Y}}
\newcommand{\bZ}{\mathbf{Z}}

% calligraphic
\newcommand{\cf}{\mathcal{f}}
%\newcommand{\cA}{\mathcal{A}}
\newcommand{\cB}{\mathcal{B}}
\newcommand{\cC}{\mathcal{C}}
%\newcommand{\cD}{\mathcal{D}}
\newcommand{\cE}{\mathcal{E}}
\newcommand{\cF}{\mathcal{F}}
\newcommand{\cG}{\mathcal{G}}
\newcommand{\cH}{\mathcal{H}}
\newcommand{\cI}{\mathcal{I}}
\newcommand{\cJ}{\mathcal{J}}
\newcommand{\cL}{\mathcal{L}}
%\newcommand{\cM}{\mathcal{M}}
\newcommand{\cN}{\mathcal{N}}
\newcommand{\cO}{\mathcal{O}}
\newcommand{\cP}{\mathcal{P}}
\newcommand{\cQ}{\mathcal{Q}}
%\newcommand{\cR}{\mathcal{R}}
%\newcommand{\cS}{\mathcal{S}}
\newcommand{\cT}{\mathcal{T}}
%\newcommand{\cU}{\mathcal{U}}
\newcommand{\cV}{\mathcal{V}}
\newcommand{\cW}{\mathcal{W}}
\newcommand{\cX}{\mathcal{X}}
\newcommand{\cY}{\mathcal{Y}}
\newcommand{\cZ}{\mathcal{Z}}

\newcommand{\rf}{\mathscr{f}}
\newcommand{\rA}{\mathscr{A}}
\newcommand{\rB}{\mathscr{B}}
\newcommand{\rC}{\mathscr{C}}
\newcommand{\rD}{\mathscr{D}}
\newcommand{\rE}{\mathscr{E}}
\newcommand{\rF}{\mathscr{F}}
\newcommand{\rG}{\mathscr{G}}
\newcommand{\rH}{\mathscr{H}}
\newcommand{\rI}{\mathscr{I}}
\newcommand{\rJ}{\mathscr{J}}
\newcommand{\rK}{\mathscr{K}}
\newcommand{\rL}{\mathscr{L}}
\newcommand{\rM}{\mathscr{M}}
\newcommand{\rN}{\mathscr{N}}
\newcommand{\rO}{\mathscr{O}}
\newcommand{\rP}{\mathscr{P}}
\newcommand{\rQ}{\mathscr{Q}}
\newcommand{\rR}{\mathscr{R}}
\newcommand{\rS}{\mathscr{S}}
\newcommand{\rT}{\mathscr{T}}
\newcommand{\rU}{\mathscr{U}}
\newcommand{\rV}{\mathscr{V}}
\newcommand{\rW}{\mathscr{W}}
\newcommand{\rX}{\mathscr{X}}
\newcommand{\rY}{\mathscr{Y}}
\newcommand{\rZ}{\mathscr{Z}}

\newcommand{\bbf}{\mathbb{f}}
\newcommand{\bbA}{\mathbb{A}}
\newcommand{\bbB}{\mathbb{B}}
\newcommand{\bbC}{\mathbb{C}}
\newcommand{\bbD}{\mathbb{D}}
\newcommand{\bbE}{\mathbb{E}}
\newcommand{\bbF}{\mathbb{F}}
\newcommand{\bbG}{\mathbb{G}}
\newcommand{\bbH}{\mathbb{H}}
\newcommand{\bbI}{\mathbb{I}}
\newcommand{\bbJ}{\mathbb{J}}
\newcommand{\bbK}{\mathbb{K}}
\newcommand{\bbL}{\mathbb{L}}
\newcommand{\bbM}{\mathbb{M}}
\newcommand{\bbN}{\mathbb{N}}
\newcommand{\bbO}{\mathbb{O}}
\newcommand{\bbP}{\mathbb{P}}
\newcommand{\bbQ}{\mathbb{Q}}
\newcommand{\bbR}{\mathbb{R}}
\newcommand{\bbS}{\mathbb{S}}
\newcommand{\bbT}{\mathbb{T}}
\newcommand{\bbU}{\mathbb{U}}
\newcommand{\bbV}{\mathbb{V}}
\newcommand{\bbW}{\mathbb{W}}
\newcommand{\bbX}{\mathbb{X}}
\newcommand{\bbY}{\mathbb{Y}}
\newcommand{\bbZ}{\mathbb{Z}}


%%%%%%%%%%%%%%%%%%%%%%%%%%%%
% Math jargon              %
%%%%%%%%%%%%%%%%%%%%%%%%%%%%
\newcommand{\wrt}{w.r.t.\xspace}
\newcommand{\defeq}{\stackrel{\mathclap{\normalfont\mbox{\scriptscriptstyle def}}}{=}}
\newcommand{\maxund}[1]{\max\limits_{#1}}
\newcommand{\supund}[1]{\text{sup}\limits_{#1}}
\newcommand{\minund}[1]{\min\limits_{#1}}
\renewcommand{\epsilon}{\varepsilon}
\newcommand{\bigotime}{\mathcal{O}}


\DeclareMathOperator*{\argmin}{arg\,min} 
\DeclareMathOperator*{\argmax}{arg\,max} 
\DeclareMathOperator*{\cupdot}{\mathbin{\mathaccent\cdot\cup}}

%%%%%%%%%%%%%%%%%%%%%%%%%%%%
% Matrix operators         %
%%%%%%%%%%%%%%%%%%%%%%%%%%%%
\newcommand{\transp}{\mathsf{\scriptscriptstyle T}}

%%%%%%%%%%%%%%%%%%%%%%%%%%%%
% Statistic operators      %
%%%%%%%%%%%%%%%%%%%%%%%%%%%%
\newcommand{\probability}[1]{\mathbb{P}\left(#1\right)}
\newcommand{\probdist}{Pr}
\DeclareMathOperator*{\expectedvalue}{\mathbb{E}}
\DeclareMathOperator*{\variance}{\text{Var}}
\newcommand{\expectedvalueover}[1]{\expectedvalue\limits_{#1}}
\newcommand{\condbar}{\;\middle|\;}
\newcommand{\gaussdistr}{\mathcal{N}}
\newcommand{\uniformdistr}{\mathcal{U}}
\newcommand{\bernoullidist}{\mathcal{B}}

%%%%%%%%%%%%%%%%%%%%%%%%%%%%
% Algebraic Sets           %
%%%%%%%%%%%%%%%%%%%%%%%%%%%%
\newcommand{\Real}{\mathbb{R}}
\newcommand{\Natural}{\mathbb{N}}
\newcommand{\statespace}{\mathcal{X}}
\newcommand{\funcspace}{\mathcal{F}}
\newcommand{\dynaspace}{\mathcal{T}}


%\newtheorem{theorem}{Theorem}
%\newtheorem{definition}{Definition}
%\newtheorem{lemma}{Lemma}
%\newtheorem{proposition}{Proposition}
%\newtheorem{remark}{Remark}
%\newtheorem{property}{Property}
%\newtheorem{assumption}{Assumption}
%\newtheorem{conjecture}{Conjecture}


%\input{../glossary.tex}
\newcommand{\Q}{Q}
\newcommand{\V}{V}
\newcommand{\mubot}{\mu_{\bot}}
\newcommand{\mutop}{\mu_{\top}}
\newcommand{\params}{\theta}
\newcommand{\dirac}{\delta}
\newcommand{\normal}{\mathcal{N}}
\newcommand{\binomial}{\mathcal{B}}
\newcommand{\features}{\phi}
\newcommand{\maxiteration}{K}
\newcommand{\deltastoppingcriterion}{\upsilon}
\newcommand{\extrasmallvalue}{\kappa}
\newcommand{\egreedy}{\epsilon}
\newcommand{\users}{\rU}
\newcommand{\timeslot}{\tau}
\newcommand{\cooperationrate}{\varrho}
\newcommand{\T}{N}
\newcommand{\srs}{\nu}
\newcommand{\ser}{\xi}
\newcommand{\transpose}{\top}
\newcommand{\indextransition}{i}
\newcommand{\state}{s}
\newcommand{\n}{k}
\newcommand{\learningrate}{\alpha}
\newcommand{\abo}{\overline{\mathcal{T}}}
\newcommand{\bo}{\mathcal{T}}
\newcommand{\oQ}{\overline{Q}}
\newcommand{\Qr}{Q_r}
\newcommand{\Qc}{Q_c}
\newcommand{\oR}{\overline{R}}
\newcommand{\oV}{\overline{V}}
\newcommand{\Vr}{V_r}
\newcommand{\Vc}{V_c}
\newcommand{\ocS}{\overline{\mathcal{S}}}
\newcommand{\ocA}{\overline{\mathcal{A}}}
\newcommand{\cS}{\mathcal{S}}
\newcommand{\budgetspace}{\mathscr{B}}
\newcommand{\cK}{\mathcal{K}}
\newcommand{\policies}{\overline{\Pi}}
\newcommand{\cU}{\mathcal{U}}
\newcommand{\cA}{\mathcal{A}}
\newcommand{\augmentedtransition}{\overline{P}}
\newcommand{\cD}{\mathcal{D}}
\newcommand{\reward}{R}
\newcommand{\augmentedreward}{\overline{R}}
\newcommand{\constraint}{C}
\newcommand{\transition}{P}
\newcommand{\return}{G}
\newcommand{\constraintreturn}{G_c}
\newcommand{\augmentedreturn}{\overline{G}}
\newcommand{\cM}{\mathcal{M}}
\newcommand{\policy}{\pi}
\newcommand{\budgetedpolicy}{\overline{\pi}}
\newcommand{\optimalpolicy}{\pi^*}
\newcommand{\optimalbudgetedpolicy}{\overline{\pi}^*}
\newcommand{\discountfactor}{\gamma}
\newcommand{\budgetaction}{\beta_a}
\newcommand{\nextbudget}{\beta'}
\newcommand{\budget}{\beta}
\newcommand{\projection}{\Xi}
\newcommand{\augmentedprojection}{\overline{\Xi}}

\beamertemplatenavigationsymbolsempty

\author[shortname]{Nicolas Carrara}
\institute[shortinst]{University of Toronto}

\title[ROMA]{ROMA: Multi-Agent RL with Emergent Roles}
\subtitle{Tonghan Wang, Heng Dong, Victor Lesser and Chongjie Zhang}

\date{April 15, 2020}
\begin{document}
    \begin{frame}
        \maketitle
        \centering
    \end{frame}

    \begin{frame}{Motivation}

        MARL deal with non-stationarity using Centralized-Training Dec-Execution (CTDE).

        \begin{alertblock}{Limit}
            Not sufficient for complex tasks.
        \end{alertblock}

        \begin{exampleblock}{Solution}
            Define and assign roles/skills to agents to fullfit a general complexe task.
            \end{exampleblock}

    \end{frame}

    \begin{frame}{Background}

        % TODO maybe put a vienne diagram

        \begin{itemize}
            \item $H(X)$: Entropy.
            \item $I(X;Y)$: Mutual Information.
            \item $I(X;Y|Z)$ Conditional mutual information.

        \end{itemize}

    \end{frame}

    \begin{frame}{The loss function}
        To define the loss, those roles properties must be fullfield:

        \begin{itemize}
            \item Dynamic;
            \item Versatile;
            \item Identifiable;
            \item Specialized.
        \end{itemize}
    \end{frame}

    \begin{frame}{Dynamic}
        Two complementary aspects:

        \begin{itemize}
            \item Roles encoding in a stochastic embedding space.
            \begin{itemize}
                \item $\rho_i \sim \normal(\mu_{\rho_i},\sigma_{\rho_i})$
            \end{itemize}
            \item Role are dictacted by a local observation.
            \begin{itemize}
                \item $(\mu_{\rho_i},\sigma_{\rho_i}) = f(o_i|\theta_{\rho})$
            \end{itemize}
        \end{itemize}

    \end{frame}

    \begin{frame}{Identifiable and versatile ($\mathcal{L}_{I}$)}

        \begin{itemize}
            \item Versatility: given an observation, roles should be diverse.
            \begin{itemize}
                \item Maximise $H(\rho_i|o_i)$
            \end{itemize}
            \item Identifiability: a trajectory (behavior) must be identifiable by a role.
            \begin{itemize}
                \item Minimise $H(\rho_i|\tau_i, o_i)$
            \end{itemize}
        \end{itemize}

        \begin{block}{}
            Equivalent to maximise $I(\tau_i;\rho_i|o_i)$.
        \end{block}
        \begin{block}{In other words}
            For a given agent, given an observation, we enforce a bijection between roles and trajectories.
        \end{block}
        \begin{alertblock}{}
            But what about roles \textbf{across} agents ?
        \end{alertblock}

    \end{frame}

    \begin{frame}{Specialization ($\mathcal{L}_{D}$)}
        All agents with similar behaviors should have similar role embeddings.

        \only<1>{
            \begin{exampleblock}{Solution}
                Maximize $I(\rho_i;\tau_j)$
            \end{exampleblock}

            \begin{alertblock}{Limit}
                All agents will have the same role (and policy)
            \end{alertblock}
        }
        \only<2>{
            \begin{exampleblock}{Solution}
                Maximize $I(\rho_i;\tau_j) + \mathcolorbox {green}{d_{\phi}(\tau_i,\tau_j)}$
            \end{exampleblock}

            \begin{alertblock}{Limit}
                Might discard $d_{\phi}$
            \end{alertblock}
        }

        \only<3>{
            \begin{exampleblock}{Solution (final)}
                Minimize $\mathcolorbox {green}{||(d_{\phi}(\tau_i,\tau_j))_{i,j}||_{0} }$\\
                s.t. $I(\rho_i;\tau_j) + {d_{\phi}(\tau_i,\tau_j)}\mathcolorbox {green}{>U}$
            \end{exampleblock}

            where $U$ is the compactness of the embedding space.

        }

    \end{frame}


%    \begin{exampleblock}{Solution}
%        Learn a dissimilarity model $d_{\phi}$ for trajectories:
%        \begin{itemize}
%            \item Maximise $I(\rho_i;\tau_j) + d_{\phi}(\tau_i,\tau_j)$
%        \end{itemize}
%    \end{exampleblock}

    \begin{frame}{Putting things togetheir}

        Previous objectives can be expressed as 3 differentiable losses.

        \vspace{0.2in}
        \only<1>{
            \begin{block}{Final loss}
                $\mathcal{L}(\theta)= \mathcal{L}_{TD}(\theta)
                + \lambda_{I} \mathcal{L}_{I}(\theta_{\rho},\xi)
                + \lambda_{D}\mathcal{L}_{D}(\theta_{\rho},\xi,\phi)$
            \end{block}
          \begin{itemize}
                \item[] {}
            \end{itemize}
        }

        \only<2>{
            \begin{block}{Final loss}
                $\mathcal{L}(\theta)= \mathcolorbox {green}{\mathcal{L}_{TD}(\theta)}
                + \lambda_{I} \mathcal{L}_{I}(\theta_{\rho},\xi)
                + \lambda_{D}\mathcal{L}_{D}(\theta_{\rho},\xi,\phi)$
            \end{block}

            \begin{itemize}
                \item Qmix Temporal difference loss.
            \end{itemize}
        }
        \only<3>{
            \begin{block}{Final loss}
                $\mathcal{L}(\theta)= \mathcal{L}_{TD}(\theta)
                + \lambda_{I} \mathcolorbox {green}{\mathcal{L}_{I}(\theta_{\rho},\xi)}
                + \lambda_{D}\mathcal{L}_{D}(\theta_{\rho},\xi,\phi)$
            \end{block}

            \begin{itemize}
                \item Identifiability and versatility loss.
            \end{itemize}
        }
        \only<4>{
            \begin{block}{Final loss}
                $\mathcal{L}(\theta)= \mathcal{L}_{TD}(\theta)
                + \lambda_{I} \mathcal{L}_{I}(\theta_{\rho},\xi)
                + \lambda_{D}\mathcolorbox {green}{\mathcal{L}_{D}(\theta_{\rho},\xi,\phi)}$
            \end{block}

            \begin{itemize}
                \item Specialization loss.
            \end{itemize}
        }
    \end{frame}

%    \section{Transfer Learning}

%    \subsection{Markov Decision Process limitations}

%    \begin{frame}
%        \begin{block}{Markov Decision Processes (MDP)}
%
%            A tuple $(\cS, \cA, P, R, \gamma)$ where
%
%            \begin{itemize}
%                \item $\cS$ is the state space; $\cA$ is the action space.
%                \item $P$ the transition function (or dynamic); $R$ the reward function.
%                \item $\gamma$ the discounted factor.
%            \end{itemize}
%        \end{block}
%
%        \pause
%        \begin{block}{Solving an MDP}
%            \begin{itemize}
%                \item $G_r^\pi = \sum_{t=0}^\infty \gamma^t R(s_t, a_t)$ return of rewards.
%                \item Find $\pi^*$ s.t $\forall s\in\cS$: $\pi^* \in \argmax_{\pi\in\cM(\cA)^\cS} \expectedvalue[G_r^\pi | s_0=s]$

%            \end{itemize}
%        \end{block}
%        \pause
%        \begin{block}{Limitations}
%            \begin{itemize}
%                \item  $\cS$ (or $\cA$) is large of continuous ? $\rightarrow$ Function approximation (FA).
%                \item  $P$ or $R$ are unknown ? $\rightarrow$ Reinforcement Learning (RL).
%                \item Too few data, learning from scratch ? $\rightarrow$ Transfer Learning (TL).
%            \end{itemize}
%        \end{block}
%    \end{frame}
%
%    \subsection{The problem}
%
%    \foreach \n in {1}{
%        \begin{frame}{}
%            \begin{figure}
%                \centering
%                %%\vspace{-1.5em}
%                \includegraphics[scale=0.75,page=1]{tl\n.pdf}
%            \end{figure}
%        \end{frame}
%    }




\end{document}

